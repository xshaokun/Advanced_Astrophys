\chapter{银河系}
\section{银河系的形态}
千亿颗恒星构成了银河系,太阳到银河系中的距离约为8\,kpc,银河系的结构整体呈现为漩涡状,其中大致可分为一下几部分:
\begin{itemize}
  \item 银盘,恒星主要分布的区域,直径约为50\,kpc。根据成分不同可以分为薄盘和厚盘:薄盘包含较年轻的恒星和星际介质,垂直标高为350\,pc;厚盘包含年龄较大的恒星,银道面上的恒星密度只有薄盘的8.5\%,垂直标高可达1000\,pc
  \item 银核,银河系中心恒星密度很大,光度很大的区域,半径可达0.7\,kpc
  \item 银棒,位于银河系中心,半径可达4\,kpc
  \item 银晕,银盘周围的球形气体,半径可达100\,kpc,其中分布着星团
  \item 暗物质晕,整个银河系都镶嵌在暗物质晕中,半径可达230\,kpc,具有银河系的极大部分质量
\end{itemize}

由于金属元素只能通过核反应合成,而铁元素主要通过超新星爆发产生,因为中小质量很无法生成合成铁,而大质量恒星只能生成铁核,核的质量毕竟很少。根据这种性质,我们可以认为恒星在演化过程中几乎不生成铁,那么类似前面的星族分类,年轻的恒星往往铁含量较高,因为是基于前代恒星的残骸形成,而年老的恒星形成很早,铁含量很低,类比到金属含量也是相同的规律,这种规律被称为\textbf{年龄-金属丰度关系}。由此我们可以通过铁来定义\textbf{金属丰度},且和恒星年龄有相关性:
\begin{equation}
  [\mathrm{Fe/H}]\equiv \log_{10}\left[{(N_\mathrm{Fe}/N_\mathrm{H})_\mathrm{star}\over (N_\mathrm{Fe}/N_\mathrm{H})_\odot}\right]
\end{equation}

上式以太阳为基准来定义金属丰度,因此太阳的$[\mathrm{Fe/H}]=0.0$。

根据引力的作用效果,考虑暗物质晕的密度分布应遵循
\begin{equation}
  \rho(r)={\rho_0 \over (r/a)(1+r/a)^2}
\end{equation}

在远离银晕的区域密度以$1/r^2$减小分布,在靠近中心处稍微平缓一点以$1/r$分布。

\section{银河系中心}
银河系的中心位于半人马座A的方向,从那个方向还能观测到X射线,其来源被认为就是银河系中心。其产生原因目前普遍认为是\textbf{银河中心存在着一个超大质量黑洞},质量$\sim 4\times 10^6\,M_\odot$。