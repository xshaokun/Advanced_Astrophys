\chapter{太阳系的小天体}
\section{彗星和柯伊伯带天体}
\paragraph{彗星}
主要成分为水冰,含少量一氧化碳冰、二氧化碳冰以及尘埃,因此彗星模型又被称为脏雪球模型。具有很大的轨道偏心率,根据周期长度可以分为以下两类:
\begin{itemize}
  \item \textbf{短周期彗星},公转周期短于200年,轨道靠近黄道面,频繁进入内太阳系,来源于柯伊伯带,哈雷彗星就是典型的短周期彗星
  \item \textbf{长周期彗星},公转周期长于200年,轨道不在黄道面上,奥尔特云
\end{itemize}

\paragraph{奥尔特云}
位于太阳系的最外围,距离太阳3,000\,AU到100,000\,AU的区域,包含了大约$10^{12}-10^{13}$个、质量约为$100\,M_\odot$的成员。奥尔特云内区(3,000\,AU到20,000\,AU)主要集中在黄道面上,外区(20,000\,AU到100,000\,AU)以球形包裹着太阳系。

\paragraph{柯伊伯带}
位于海王星以外,距离太阳30\,AU到50\,AU的区域,是位于黄道面内包含大量彗核的盘。冥王星是第一颗被发现的\textbf{伊伯带天体}。

\section{小行星}
太阳系内大部分小行星位于火星和木星轨道之间的小行星带内,是没有形成行星的星子。

小行星带中的小行星大小不等,最大的三颗小行星分别是智神星、婚神星和灶神星,平均直径超过400\,km;仅有的一颗矮行星—谷神星,直径约为950\,km。

小行星带中有一条密度较低的区域被称为柯克伍德空隙,是由于3:1的轨道共振所产生的。

\paragraph{特洛伊小行星}
还有一部分小行星位于木星轨道,与木星和太阳成等边三角形的区域。由于该区域是木星和太阳的拉格朗日点$L_3$和$L_4$,是引力势能最低点,因此小行星能够稳定在这里绕太阳公转。
