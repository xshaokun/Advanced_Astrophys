\chapter{星系演化}
\section{星系的相互作用}
当两个星系相互``碰撞''时,实际上由于恒星间的距离较大,并不会发生真正的碰撞,但是引力的作用还是会发生。两个彼此擦肩而过的恒星之间的引力作用,会有把它们拉向彼此的趋势,这会造成恒星的运动速度会下降,而在更大的尺度上,这种普遍的速度下降会使星系的运动速度也下降。这种阻碍运动效果有点类似于摩擦力,因此被称为\textbf{动力学摩擦}:
\begin{equation}
  f_d\simeq C{G^2M^2\rho \over v_M^2}
\end{equation}

其中$C$是一个无量纲量,是星系速度$v_M$和速度弥散$\sigma$的比值。

\section{星系的形成}
\subsection{ELS塌缩模型}
过程和恒星形成类似,但是是尺度更大的星云快速塌缩形成的,这能够比较好的解释许多银晕中的年老恒星具有很高偏心率的椭圆轨道,因为它们在塌缩过程中就开始形成了,还保留了很大的塌缩带来的径向速度。

但是观测上有许多这种模型无法解释的现象
\begin{itemize}
  \item 一半外银晕的恒星具有逆行轨道,因此银晕的整体角速度接近0,与模型预测不符
  \item 球状星团和银晕年龄相差太大
  \item 球状星团的化学组成不统一
\end{itemize}

因此人们提出了一些新的模型来解释观测现象:
\paragraph{耗散塌缩模型}
考虑冷却时标比自由落体时标长,塌缩过程中温度会上升,塌缩速度会变慢

\paragraph{阶梯式并合模型}
通过吞并小星系成长,这会导致不同星系的成分会混合,这能够解释球状星团年龄和金属丰度的差异。近期的高红移观测支持了这一模型,他们的观测发现高红移区(代表很久之前)具有比现在更多数量的小星系。

